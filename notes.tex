\documentclass{notes}
\usetikzlibrary{shapes, shapes.geometric}

\class{CS 181 (Introduction to Formal Languages and Automata Theory)}

\begin{document}

\section{Deterministic finite automata (DFAs)}

\subsection{Basic notions}

\begin{defn}
  An \textbf{alphabet} is any finite set of symbols.
\end{defn}

\begin{eg}
  Binary alphabet: $\left \{ \verb~0~, \verb~1~ \right \}$
\end{eg}

\begin{eg}
  English alphabet: $\left \{ \verb~a~, \verb~b~, \dots, \verb~c~ \right \}$
\end{eg}

\begin{defn}
  A \textbf{string} is any finite sequence of symbols from a given alphabet.
\end{defn}

\begin{eg}
  \verb~001010110101~
\end{eg}

\begin{eg}
  \verb~abracadabra~
\end{eg}

\begin{eg}
  $\varepsilon$ (empty string)
\end{eg}

\begin{defn}
  A \textbf{language} is a set of strings over a given alphabet.
\end{defn}

\begin{eg}
  $\varnothing$ (empty language)
\end{eg}

\begin{eg}
  $\left \{ \varepsilon \right \}$
\end{eg}

\begin{eg}
  $\left \{ \verb~acclaim~, \verb~aim~, \verb~brim~, \dots \right \}$
\end{eg}

\begin{eg}
  $\left \{ \verb~0~, \verb~1~, \verb~00~, \verb~11~, \dots \right \}$
\end{eg}

\begin{defn}
  A \textbf{computational device} is a mechanism that inputs a string and either accepts or rejects it.
\end{defn}

\subsection{Deterministic finite automata}

\begin{itemize}
  \item Choose an alphabet: $\left \{ \verb~a~, \verb~b~ \right \}$.
  \item Draw states.
  \item Choose start state and accept states.
  \item Draw transitions (out of every state on every symbol).
\end{itemize}

\begin{minipage}{0.5 \textwidth}
  \begin{center}
    \begin{tikzpicture}[> = stealth, node distance = 3em]
      \node[initial, initial text=, state, minimum size = 2em] (q1) {};
      \node[below left = of q1, accepting, state, minimum size = 2em] (q2) {};
      \node[below right = of q1, accepting, state, minimum size = 2em] (q3) {};
      \node[below right = of q2, state, minimum size = 2em] (q4) {};
      \draw[->]
      (q1) edge[above left] node{\verb~a~} (q2)
      (q1) edge[above right] node{\verb~b~} (q3)
      (q2) edge[loop left] node{\verb~a~} (q2)
      (q2) edge[below left] node{\verb~b~} (q4)
      (q3) edge[below right] node{\verb~a~} (q4)
      (q3) edge[loop right] node{\verb~b~} (q3)
      (q4) edge[loop below] node{\verb~a~,\verb~b~} (q4)
      ;
    \end{tikzpicture}
  \end{center}
\end{minipage}%
\begin{minipage}{0.5 \textwidth}
  \begin{center}
    \begin{tabular}{cc}
      Input & Output \\ 
      \hline
      $\varepsilon$ & reject \\ 
      \verb~ab~ & reject \\ 
      \verb~aaa~ & accept \\ 
      \verb~bb~ & accept
    \end{tabular}
  \end{center}
\end{minipage}

In words, this machine accepts nonempty strings of all \verb~a~'s or all \verb~b~'s.

\begin{defn}
  The \textbf{language} of a DFA is the set of all strings it accepts.
\end{defn}

\begin{eg}
  
  \begin{minipage}{0.5 \textwidth}
    \begin{center}
      \begin{tikzpicture}[> = stealth, node distance = 3em]
        \node[accepting, initial, initial text=, state, minimum size = 2em] (q1) at (90 : 1.5) {};
        \node[state, minimum size = 2em] (q2) at (330 : 1.5) {};
        \node[state, minimum size = 2em] (q3) at (210 : 1.5) {};
        \draw[->]
        (q1) edge[loop above] node{\verb~0~} (q1)
        (q1) edge[below left] node{\verb~1~} (q2)
        (q1) edge[bend right, above left] node{\verb~2~} (q3)
        (q2) edge[loop right] node{\verb~0~} (q2)
        (q2) edge[above] node{\verb~1~} (q3)
        (q2) edge[bend right, above right] node{\verb~2~} (q1)
        (q3) edge[loop left] node{\verb~0~} (q3)
        (q3) edge[below right] node{\verb~1~} (q1)
        (q3) edge[bend right, below] node{\verb~2~} (q2)
        ;
      \end{tikzpicture}
    \end{center}
  \end{minipage}%
  \begin{minipage}{0.5 \textwidth}
    \begin{center}
      \begin{tabular}{cc}
        Input & Output \\ 
        \hline
        \verb~00...0~ & accept \\ 
        \verb~12~ & accept \\ 
        \verb~111~ & accept \\ 
        \verb~20~ & reject \\ 
        \verb~1~ & reject
      \end{tabular}
    \end{center}
  \end{minipage}

  \begin{center}
    Alphabet: $\left \{ \verb~0~, \verb~1~, \verb~2~ \right \}$, language: $\left \{ w : 3 \mid \sum w_i \right \}$
  \end{center}
\end{eg}

\begin{eg}
  \begin{center}
    \begin{tikzpicture}[> = stealth, node distance = 3em]
      \node[accepting, initial, initial text=, state, minimum size = 2em] (q1) {};
      \node[right = of q1, state, minimum size = 2em] (q2) {};
      \draw[->]
      (q1) edge[bend right, below] node{\verb~0~,\verb~1~} (q2)
      (q2) edge[bend right, above] node{\verb~0~,\verb~1~} (q1)
      ;
    \end{tikzpicture}

    Alphabet: $\left \{ \verb~0~, \verb~1~ \right \}$, language: $\left \{ w : 2 \mid \left | w \right | \right \}$
  \end{center}
\end{eg}

\begin{eg}
  \begin{center}
    \begin{tikzpicture}[> = stealth, node distance = 3em]
      \node[initial, initial text=, state, minimum size = 2em] (q1) {};
      \node[accepting, below left = of q1, state, minimum size = 2em] (q2) {};
      \node[below = of q2, state, minimum size = 2em] (q3) {};
      \node[accepting, below right = of q1, state, minimum size = 2em] (q4) {};
      \node[below = of q4, state, minimum size = 2em] (q5) {};
      \draw[->]
      (q1) edge[above left] node{\verb~a~} (q2)
      (q1) edge[above right] node{\verb~b~} (q4)
      (q2) edge[loop left] node{\verb~a~} (q2)
      (q2) edge[bend right, left] node{\verb~b~} (q3)
      (q3) edge[bend right, right] node{\verb~a~} (q2)
      (q3) edge[loop below] node{\verb~b~} (q3)
      (q4) edge[bend right, left] node{\verb~a~} (q5)
      (q4) edge[loop right] node{\verb~b~} (q4)
      (q5) edge[loop below] node{\verb~a~} (q5)
      (q5) edge[bend right, right] node{\verb~b~} (q4)
      ;
    \end{tikzpicture}

    Alphabet: $\left \{ \verb~a~, \verb~b~ \right \}$, language: $\left \{ w : w \neq \varepsilon \land w_1 = w_{\left | w \right |  } \right \}$
  \end{center}
\end{eg}

\subsection{Designing DFAs}

We will be using the binary alphabet $\left \{ \verb~0~, \verb~1~ \right \}$.

\begin{eg}
  Language: $\varnothing$
  
  \begin{center}
    \begin{tikzpicture}[> = stealth, node distance = 3em]
      \node[initial, initial text=, state, minimum size = 2em] (q1) {};
      \draw[->]
      (q1) edge[loop right] node{\verb~0~,\verb~1~} (q1)
      ;
    \end{tikzpicture}
  \end{center}
\end{eg}

\begin{eg}
  Language: $\left \{ w : \text{every odd position is a \verb~1~} \right \}$
  
  \begin{center}
    \begin{tikzpicture}[> = stealth, node distance = 3em]
      \node[accepting, initial, initial text=, state, minimum size = 2em] (q1) {};
      \node[accepting, below right = of q1, state, minimum size = 2em] (q2) {};
      \node[above right = of q1, state, minimum size = 2em] (q3) {};
      \draw[->]
      (q1) edge[above left] node{\verb~0~} (q3)
      (q1) edge[bend right, below left] node{\verb~1~} (q2)
      (q2) edge[bend right, above right] node{\verb~0~,\verb~1~} (q1)
      (q3) edge[loop right] node{\verb~0~,\verb~1~} (q3)
      ;
    \end{tikzpicture}
  \end{center}
\end{eg}

\begin{eg}
  Language: $\left \{ w : \text{$w$ ends in \verb~0~} \right \}$
  
  \begin{center}
    \begin{tikzpicture}[> = stealth, node distance = 3em]
      \node[initial, initial text=, state, minimum size = 2em] (q1) {};
      \node[right = of q1, accepting, state, minimum size = 2em] (q2) {};
      \draw[->]
      (q1) edge[bend right, below] node{\verb~0~} (q2)
      (q1) edge[loop above] node{\verb~1~} (q1)
      (q2) edge[loop below] node{\verb~0~} (q2)
      (q2) edge[bend right, above] node{\verb~1~} (q1)
      ;
    \end{tikzpicture}
  \end{center}
\end{eg}

\begin{eg}
  Language: $\left \{ w : \text{$w$ begins with \verb~1~, ends with \verb~0~} \right \}$
  
  \begin{center}
    \begin{tikzpicture}[> = stealth, node distance = 3em]
      \node[initial, initial text=, state, minimum size = 2em] (q1) {};
      \node[above right = of q1, state, minimum size = 2em] (q2) {};
      \node[right = of q2, accepting, state, minimum size = 2em] (q3) {};
      \node[below right = of q1, state, minimum size = 2em] (q4) {};
      \draw[->]
      (q1) edge[below left] node{\verb~0~} (q4)
      (q1) edge[above left] node{\verb~1~} (q2)
      (q2) edge[bend right, below] node{\verb~0~} (q3)
      (q2) edge[loop above] node{\verb~1~} (q2)
      (q3) edge[loop right] node{\verb~0~} (q3)
      (q3) edge[bend right, above] node{\verb~1~} (q2)
      (q4) edge[loop right] node{\verb~0~,\verb~1~} (q4)
      ;
    \end{tikzpicture}
  \end{center}
\end{eg}

\begin{eg}
  Language: $\left \{ w : \left | w \right | \leq 4  \right \}$
  
  \begin{center}
    \begin{tikzpicture}[> = stealth, node distance = 3em]
      \node[accepting, initial, initial text=, state, minimum size = 2em] (q1) {};
      \node[right = of q1, accepting, state, minimum size = 2em] (q2) {};
      \node[accepting, right = of q2, state, minimum size = 2em] (q3) {};
      \node[accepting, right = of q3, state, minimum size = 2em] (q4) {};
      \node[accepting, right = of q4, state, minimum size = 2em] (q5) {};
      \node[right = of q5, state, minimum size = 2em] (q6) {};
      \draw[->]
      (q1) edge[above] node{\verb~0~,\verb~1~} (q2)
      (q2) edge[above] node{\verb~0~,\verb~1~} (q3)
      (q3) edge[above] node{\verb~0~,\verb~1~} (q4)
      (q4) edge[above] node{\verb~0~,\verb~1~} (q5)
      (q5) edge[above] node{\verb~0~,\verb~1~} (q6)
      (q6) edge[loop right] node{\verb~0~,\verb~1~} (q6)
      ;
    \end{tikzpicture}
  \end{center}
\end{eg}

\newpage

\begin{eg}
  Language: $\left \{ w : 1000 \mid \left | w \right | \right \}$
  
  In words, each state represents a remainder modulo 1000, and only the 0 state is accepting.
\end{eg}

\begin{eg}
  Language: $\left \{ w : \text{$w$ contains \verb~0101~ as a substring} \right \}$
  
  \begin{center}
    \begin{tikzpicture}[> = stealth, node distance = 3em]
      \node[initial, initial text=, state, minimum size = 2em] (q1) {$\varepsilon$};
      \node[right = of q1, state, minimum size = 2em] (q2) {\verb~0~};
      \node[right = of q2, state, minimum size = 2em] (q3) {\verb~01~};
      \node[right = of q3, state, minimum size = 2em] (q4) {\verb~010~};
      \node[accepting, right = of q4, state, minimum size = 2em] (q5) {\verb~0101~};
      \draw[->]
      (q1) edge[above] node{\verb~0~} (q2)
      (q1) edge[loop above] node{\verb~1~} (q1)
      (q2) edge[loop above] node{\verb~0~} (q2)
      (q2) edge[above] node{\verb~1~} (q3)
      (q3) edge[above] node{\verb~0~} (q4)
      (q3) edge[bend right = 75, above] node{\verb~1~} (q1)
      (q4) edge[bend right = 60, above] node{\verb~0~} (q2)
      (q4) edge[above] node{\verb~1~} (q5)
      (q5) edge[loop right] node{\verb~0~,\verb~1~} (q5)
      ;
    \end{tikzpicture}
  \end{center}
\end{eg}

\begin{eg}[Week 1 Discussion]
  $L = \left \{ w : \left | w \right | > 0 \land \text{$w$ contains only \verb~1~s} \right \}$
  
  \begin{center}
    \begin{tikzpicture}[> = stealth, node distance = 3em]
      \node[initial, initial text=, state, minimum size = 2em] (q1) {};
      \node[right = of q1, accepting, state, minimum size = 2em] (q2) {};
      \node[right = of q2, state, minimum size = 2em] (q3) {};
      \draw[->]
      (q1) edge[bend right, below] node{\verb~0~} (q3)
      (q1) edge[above] node{\verb~1~} (q2)
      (q2) edge[above] node{\verb~0~} (q3)
      (q2) edge[loop above] node{\verb~1~} (q2)
      (q3) edge[loop right] node{\verb~0~,\verb~1~} (q3)
      ;
    \end{tikzpicture}
  \end{center}
\end{eg}

\begin{eg}[Week 1 Discussion]
  $L = \left \{ w : \text{$w$ ends in \verb~1101~} \right \}$
  
  \begin{center}
    \begin{tikzpicture}[> = stealth, node distance = 3em]
      \node[initial, initial text=, state, minimum size = 2em] (q1) {};
      \node[right = of q1, state, minimum size = 2em] (q2) {};
      \node[right = of q2, state, minimum size = 2em] (q3) {};
      \node[right = of q3, state, minimum size = 2em] (q4) {};
      \node[accepting, right = of q4, state, minimum size = 2em] (q5) {};
      \draw[->]
      (q1) edge[loop below] node{\verb~0~} (q1)
      (q1) edge[bend right, below] node{\verb~1~} (q2)
      (q2) edge[bend right, above] node{\verb~0~} (q1)
      (q2) edge[above] node{\verb~1~} (q3)
      (q3) edge[above] node{\verb~0~} (q4)
      (q3) edge[loop above] node{\verb~1~} (q3)
      (q4) edge[bend right = 60, above] node{\verb~0~} (q1)
      (q4) edge[above] node{\verb~1~} (q5)
      (q5) edge[bend right = 75, above] node{\verb~0~} (q1)
      (q5) edge[bend right, above] node{\verb~1~} (q3)
      ;
    \end{tikzpicture}
  \end{center}
\end{eg}

\begin{eg}[Week 1 Discussion]
  $L = \left \{ w : \text{$w$ contains \verb~1101~} \right \}$
  
  \begin{center}
    \begin{tikzpicture}[> = stealth, node distance = 3em]
      \node[initial, initial text=, state, minimum size = 2em] (q1) {};
      \node[right = of q1, state, minimum size = 2em] (q2) {};
      \node[right = of q2, state, minimum size = 2em] (q3) {};
      \node[right = of q3, state, minimum size = 2em] (q4) {};
      \node[accepting, right = of q4, state, minimum size = 2em] (q5) {};
      \draw[->]
      (q1) edge[loop below] node{\verb~0~} (q1)
      (q1) edge[bend right, below] node{\verb~1~} (q2)
      (q2) edge[bend right, above] node{\verb~0~} (q1)
      (q2) edge[above] node{\verb~1~} (q3)
      (q3) edge[above] node{\verb~0~} (q4)
      (q3) edge[loop above] node{\verb~1~} (q3)
      (q4) edge[bend right = 60, above] node{\verb~0~} (q1)
      (q4) edge[above] node{\verb~1~} (q5)
      (q5) edge[loop right] node{\verb~0~,\verb~1~} (q5)
      ;
    \end{tikzpicture}
  \end{center}
\end{eg}

\newpage

\subsection{Formal definitions}

\begin{defn}
  A DFA is a tuple $(Q, \Sigma, \delta, q_{0}, F)$
  where
  \begin{itemize}
    \item $Q = \text{set of states, }$
    \item $\Sigma = \text{alphabet, }$
    \item $\delta = \text{transition ($\delta \colon Q \times \Sigma \to Q$), }$
    \item $q_{0} = \text{start state ($q_{0} \in Q$), and }$
    \item $F = \text{set of accept states ("favorable"? states, $F \subseteq Q$).}$
  \end{itemize}
\end{defn}

\begin{eg}
  \begin{center}
    \begin{tikzpicture}[> = stealth, node distance = 3em]
      \node[initial, initial text=, state, minimum size = 2em] (q1) {$A$};
      \node[accepting, right = of q1, accepting, state, minimum size = 2em] (q2) {$B$};
      \node[right = of q2, state, minimum size = 2em] (q3) {$C$};
      \draw[->]
      (q1) edge[loop above] node{\verb~0~} (q1)
      (q1) edge[above] node{\verb~1~} (q2)
      (q2) edge[bend right, below] node{\verb~0~} (q3)
      (q2) edge[loop above] node{\verb~1~} (q2)
      (q3) edge[bend right, above] node{\verb~0~,\verb~1~} (q2)
      ;
    \end{tikzpicture}
  \end{center}
  
  Formal description: $(\left \{ A, B, C \right \}, \left \{ \verb~0~, \verb~1~ \right \}, \delta, A, \left \{ B \right \})$ where $\delta$ is defined by the table
  \begin{center}
    \begin{tabular}{c|cc}
      & \verb~0~ & \verb~1~ \\ 
      \hline
      $A$ & $A$ & $B$ \\ 
      $B$ & $C$ & $B$ \\ 
      $C$ & $B$ & $B$.
    \end{tabular}
  \end{center}
\end{eg}

\begin{eg}
  Formal description: $(\left \{ A, B, C, D, E \right \}, \left \{ \verb~0~, \verb~1~ \right \}, \delta, C, \left \{ C \right \})$ where $\delta$ is defined by the table \begin{center}
    \begin{tabular}{c|cc}
      & \verb~0~ & \verb~1~ \\ 
      \hline
      $A$ & $A$ & $B$ \\ 
      $B$ & $A$ & $C$ \\ 
      $C$ & $B$ & $D$ \\ 
      $D$ & $C$ & $E$ \\ 
      $E$ & $D$ & $E$.
    \end{tabular}
  \end{center}
  
  \begin{center}
    \begin{tikzpicture}[> = stealth, node distance = 3em]
      \node[state, minimum size = 2em] (q1) {$A$};
      \node[right = of q1, state, minimum size = 2em] (q2) {$B$};
      \node[accepting, initial above, initial text=, right = of q2, state, minimum size = 2em] (q3) {$C$};
      \node[right = of q3, state, minimum size = 2em] (q4) {$D$};
      \node[right = of q4, state, minimum size = 2em] (q5) {$E$};
      \draw[->]
      (q1) edge[loop left] node{\verb~0~} (q1)
      (q1) edge[bend right, below] node{\verb~1~} (q2)
      (q2) edge[bend right, above] node{\verb~0~} (q1)
      (q2) edge[bend right, below] node{\verb~1~} (q3)
      (q3) edge[bend right, above] node{\verb~0~} (q2)
      (q3) edge[bend right, below] node{\verb~1~} (q4)
      (q4) edge[bend right, above] node{\verb~0~} (q3)
      (q4) edge[bend right, below] node{\verb~1~} (q5)
      (q5) edge[bend right, above] node{\verb~0~} (q4)
      (q5) edge[loop right] node{\verb~1~} (q5)
      ;
    \end{tikzpicture}
  \end{center}
\end{eg}

\begin{eg}
  Formal description for Example~\ref{eg:1.3.6}: $(\left \{ 0, 1, 2, \dots, 999 \right \}, \left \{ \verb~0~, \verb~1~ \right \}, \delta, 0, \left \{ 0 \right \})$ where $\delta(q, \sigma) = (q + 1) \mod 1000$.
\end{eg}

\newpage

\begin{defn}
  DFA $(Q, \Sigma, \delta, q_{0}, F)$ {\boldmath \bfseries accepts} a string $w = w_1 w_2 \dots w_n$ iff
  \[
    \delta(\cdots \delta(\delta(q_0, w_1), w_2) \cdots, w_n) \in F.
  \]
\end{defn}

\begin{defn}
  DFA $D$ {\boldmath \bfseries recognizes} the language $\mathcal L$ iff 
  \[
    \mathcal L = \left \{ w : \text{$D$ accepts $w$} \right \}.
  \]
\end{defn}

\begin{note}
  \begin{itemize}
    \item Every DFA recognizes exactly 1 language.

    \item A language has either 0 or $\infty$ DFAs recognizing it.
  \end{itemize}
\end{note}

\section{Nondeterminism}

\subsection{Basic notions}

\begin{eg}
  \begin{center}
    \begin{tikzpicture}[> = stealth, node distance = 3em]
      \node[initial, initial text=, state, minimum size = 2em] (q1) {$A$};
      \node[right = of q1, state, minimum size = 2em] (q2) {$B$};
      \node[right = of q2, state, minimum size = 2em] (q3) {$C$};
      \node[accepting, right = of q3, state, minimum size = 2em] (q4) {$D$};
      \draw[->]
      (q1) edge[loop above] node{\verb~0~,\verb~1~} (q1)
      (q1) edge[above] node{\verb~1~} (q2)
      (q2) edge[above] node{\verb~0~,$\varepsilon$} (q3)
      (q3) edge[above] node{\verb~1~} (q4)
      (q4) edge[loop above] node{\verb~0~,\verb~1~} (q4)
      ;
    \end{tikzpicture}
  \end{center}

  \begin{itemize}
    \item Choose an alphabet: $\left \{ \verb~0~, \verb~1~ \right \}$.

    \item Draw states.
      
    \item Choose start state and accept states.
    The steps so far are the same as those of a DFA.
      
    \item Draw transitions.
    A state may have any number of transitions on a given symbol.
    A state may also have transitions on $\varepsilon$.
  \end{itemize}
\end{eg}

\begin{defn}
  An NFA {\boldmath \bfseries accepts} $w$ iff there is \textit{at least} one path to an accept state.
\end{defn}

\newpage

\begin{eg}
  Output table for Example~\ref{eg:2.1.1}:
  \begin{center}
    \begin{tabular}{ccc}
      Input & Accepting path & Output \\ 
      \hline
      $\varepsilon$ & - & reject \\ 
      \verb~0~ & - & reject \\ 
      \verb~1~ & - & reject \\ 
      \verb~010110~ & $AABCDDD$ & accept \\
      \verb~010~ & - & reject \\ 
      \verb~11~ & $ABCD$ & accept 
    \end{tabular}
  \end{center}
  
  \begin{center}
    Language: all strings containing \verb~101~ or \verb~11~
  \end{center}
\end{eg}

\subsection{Using shortcuts}

\begin{eg}
  Language: $\varnothing$
  
  \begin{center}
    \begin{tikzpicture}[> = stealth, node distance = 3em]
      \node[initial, initial text=, state, minimum size = 2em] (q1) {};
      \draw[->]
      ;
    \end{tikzpicture}
  \end{center}
\end{eg}

\begin{eg}
  Language: $\left \{ \varepsilon \right \}$

  \begin{center}
    \begin{tikzpicture}[> = stealth, node distance = 3em]
      \node[accepting, initial, initial text=, state, minimum size = 2em] (q1) {};
      \draw[->]
      ;
    \end{tikzpicture}
  \end{center}
\end{eg}

\begin{eg}
  Language: $\left \{ w : \text{$w$ doesn't contain \verb~1~} \right \}$

  \begin{center}
    \begin{tikzpicture}[> = stealth, node distance = 3em]
      \node[accepting, initial, initial text=, state, minimum size = 2em] (q1) {};
      \draw[->]
      (q1) edge[loop right] node{\verb~0~} (q1)
      ;
    \end{tikzpicture}
  \end{center}
\end{eg}

\begin{eg}
  Language: $\left \{ w : \text{$\left | w \right | \geq 2$ and $w$ starts and ends with \verb~0~} \right \}$
  
  \begin{center}
    \begin{tikzpicture}[> = stealth, node distance = 3em]
      \node[initial, initial text=, state, minimum size = 2em] (q1) {};
      \node[right = of q1, state, minimum size = 2em] (q2) {};
      \node[accepting, right = of q2, state, minimum size = 2em] (q3) {};
      \draw[->]
      (q1) edge[above] node{\verb~0~} (q2)
      (q2) edge[above] node{\verb~0~} (q3)
      (q2) edge[loop above] node{\verb~0~,\verb~1~} (q2)
      ;
    \end{tikzpicture}
  \end{center}
\end{eg}

\subsection{Pattern matching}

\begin{eg}
  Language: $\left \{ w : \text{conatins \verb~0101~} \right \}$
  
  \begin{center}
    \begin{tikzpicture}[> = stealth, node distance = 3em]
      \node[initial, initial text=, state, minimum size = 2em] (q1) {};
      \node[right = of q1, state, minimum size = 2em] (q2) {};
      \node[right = of q2, state, minimum size = 2em] (q3) {};
      \node[right = of q3, state, minimum size = 2em] (q4) {};
      \node[accepting, right = of q4, state, minimum size = 2em] (q5) {};
      \draw[->]
      (q1) edge[loop above] node{\verb~0~,\verb~1~} (q1)
      (q1) edge[above] node{\verb~0~} (q2)
      (q2) edge[above] node{\verb~1~} (q3)
      (q3) edge[above] node{\verb~0~} (q4)
      (q4) edge[above] node{\verb~1~} (q5)
      ;
    \end{tikzpicture}
  \end{center}
\end{eg}

\newpage

\begin{eg}
  Language: $\{ w : w = \underbrace{\verb~00...0~}_{\geq 0}\underbrace{\verb~11...1~}_{\geq 0}\underbrace{\verb~00...0~}_{\geq 1} \}$ % >= 0, >= 0, >= 1
  
  \begin{center}
    \begin{tikzpicture}[> = stealth, node distance = 3em]
      \node[initial, initial text=, state, minimum size = 2em] (q1) {};
      \node[right = of q1, state, minimum size = 2em] (q2) {};
      \node[accepting, right = of q2, state, minimum size = 2em] (q3) {};
      \draw[->]
      (q1) edge[loop above] node{\verb~0~} (q1)
      (q1) edge[above] node{$\varepsilon$} (q2)
      (q2) edge[loop above] node{\verb~1~} (q2)
      (q2) edge[above] node{\verb~0~} (q3)
      (q3) edge[loop above] node{\verb~0~} (q3)
      ;
    \end{tikzpicture}
  \end{center}
\end{eg}

\begin{eg}
  Language: $\left \{ w : \text{$w$ has a \verb~1~ in the 3rd position from the end} \right \}$
  
  \begin{center}
    \begin{tikzpicture}[> = stealth, node distance = 3em]
      \node[initial, initial text=, state, minimum size = 2em] (q1) {};
      \node[right = of q1, state, minimum size = 2em] (q2) {};
      \node[right = of q2, state, minimum size = 2em] (q3) {};
      \node[accepting, right = of q3, state, minimum size = 2em] (q4) {};
      \draw[->]
      (q1) edge[loop above] node{\verb~0~,\verb~1~} (q1)
      (q1) edge[above] node{\verb~1~} (q2)
      (q2) edge[above] node{\verb~0~,\verb~1~} (q3)
      (q3) edge[above] node{\verb~0~,\verb~1~} (q4)
      ;
    \end{tikzpicture}
  \end{center}
\end{eg}

\begin{eg}[Week 1 Discussion]
  $L = \left \{ w : \text{$w$ contains \verb~1101~} \right \}$
  
  \begin{center}
    \begin{tikzpicture}[> = stealth, node distance = 3em]
      \node[initial, initial text=, state, minimum size = 2em] (q1) {};
      \node[right = of q1, state, minimum size = 2em] (q2) {};
      \node[right = of q2, state, minimum size = 2em] (q3) {};
      \node[right = of q3, state, minimum size = 2em] (q4) {};
      \node[accepting, right = of q4, state, minimum size = 2em] (q5) {};
      \draw[->]
      (q1) edge[loop above] node{\verb~0~,\verb~1~} (q1)
      (q1) edge[above] node{\verb~1~} (q2)
      (q2) edge[above] node{\verb~1~} (q3)
      (q3) edge[above] node{\verb~0~} (q4)
      (q4) edge[above] node{\verb~1~} (q5)
      (q5) edge[loop above] node{\verb~0~,\verb~1~} (q5)
      ;
    \end{tikzpicture}
  \end{center}
\end{eg}

\begin{eg}[Week 1 Discussion]
  $L = \{ w : w = \underbrace{\verb~11...1~}_{\geq 0}\underbrace{\verb~00...0~}_{\geq 1}\underbrace{\verb~11...1~}_{\geq 0} \}$
  
  \begin{center}
    \begin{tikzpicture}[> = stealth, node distance = 3em]
      \node[initial, initial text=, state, minimum size = 2em] (q1) {};
      \node[right = of q1, state, minimum size = 2em] (q2) {};
      \node[accepting, right = of q2, state, minimum size = 2em] (q3) {};
      \draw[->]
      (q1) edge[loop above] node{\verb~1~} (q1)
      (q1) edge[above] node{$\varepsilon$} (q2)
      (q2) edge[loop above] node{\verb~0~} (q2)
      (q2) edge[above] node{\verb~0~} (q3)
      (q3) edge[loop above] node{\verb~1~} (q3)
      ;
    \end{tikzpicture} 
  \end{center}
\end{eg}

\subsection{Alternatives}

\begin{eg}
  Language: $\left \{ w : 2 \mid \left | w \right | \lor 3 \mid \left | w \right | \right \}$
  
  \begin{center}
    \begin{tikzpicture}[> = stealth, node distance = 3em]
      \node[initial, initial text=, state, minimum size = 2em] (q1) {};
      \node[accepting, above right = of q1, state, minimum size = 2em] (q2) {};
      \node[right = of q2, state, minimum size = 2em] (q3) {};
      \node[accepting, below right = of q1, state, minimum size = 2em] (q4) {};
      \node[above right = of q4, state, minimum size = 2em] (q5) {};
      \node[below right = of q4, state, minimum size = 2em] (q6) {};
      \draw[->]
      (q1) edge[above left] node{$\varepsilon$} (q2)
      (q2) edge[bend right, below] node{\verb~0~,\verb~1~} (q3)
      (q3) edge[bend right, above] node{\verb~0~,\verb~1~} (q2)
      (q1) edge[below left] node{$\varepsilon$} (q4)
      (q4) edge[above left] node{\verb~0~,\verb~1~} (q5)
      (q5) edge[right] node{\verb~0~,\verb~1~} (q6)
      (q6) edge[below left] node{\verb~0~,\verb~1~} (q4)
      ;
    \end{tikzpicture}
  \end{center}
  
  \newpage
  
  Note that the following is not valid due to side effects: 

  \begin{center}
    \begin{tikzpicture}[> = stealth, node distance = 3em]
      \node[accepting, initial, initial text=, state, minimum size = 2em] (q1) {};
      \node[right = of q1, state, minimum size = 2em] (q2) {};
      \node[accepting, below = of q1, state, minimum size = 2em] (q3) {};
      \node[below left = of q3, state, minimum size = 2em] (q4) {};
      \node[below right = of q3, state, minimum size = 2em] (q5) {};
      \draw[->]
      (q1) edge[bend right, below] node{\verb~0~,\verb~1~} (q2)
      (q2) edge[bend right, above] node{\verb~0~,\verb~1~} (q1)
      (q1) edge[left] node{$\varepsilon$} (q3)
      (q3) edge[above left] node{\verb~0~,\verb~1~} (q4)
      (q4) edge[below] node{\verb~0~,\verb~1~} (q5)
      (q5) edge[above right] node{\verb~0~,\verb~1~} (q3)
      ;
    \end{tikzpicture}
  \end{center}
\end{eg}

\begin{eg}[Week 1 Discussion]
  $L = \{ w : \text{$w$ contains \verb~1101~} \lor w = \underbrace{\verb~11...1~}_{\geq 0}\underbrace{\verb~00...0~}_{\geq 1}\underbrace{\verb~11...1~}_{\geq 0} \}$
  
  \begin{center}
    \begin{tikzpicture}[> = stealth, node distance = 3em]
      \node[initial, initial text=, state, minimum size = 2em] (q1) {};
      \node[above right = of q1, style = {draw, rectangle, minimum width = 4em, minimum height = 2em}, minimum size = 2em] (q2) {NFA from Example~\ref{eg:2.3.4}};
      \node[below right = of q1, style = {draw, rectangle, minimum width = 4em, minimum height = 2em}, minimum size = 2em] (q3) {NFA from Example~\ref{eg:2.3.5}};
      \draw[->]
      (q1) edge[above left] node{$\varepsilon$} (q2.west)
      (q1) edge[below left] node{$\varepsilon$} (q3.west)
      ;
    \end{tikzpicture}
  \end{center}
\end{eg}

\end{document}